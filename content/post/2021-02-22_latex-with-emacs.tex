% Created 2021-03-15 周一 23:00
% Intended LaTeX compiler: pdflatex
\documentclass[11pt]{article}
\usepackage[utf8]{inputenc}
\usepackage[T1]{fontenc}
\usepackage{graphicx}
\usepackage{grffile}
\usepackage{longtable}
\usepackage{wrapfig}
\usepackage{rotating}
\usepackage[normalem]{ulem}
\usepackage{amsmath}
\usepackage{textcomp}
\usepackage{amssymb}
\usepackage{capt-of}
\usepackage{hyperref}
\date{\today}
\title{}
\hypersetup{
 pdfauthor={},
 pdftitle={},
 pdfkeywords={},
 pdfsubject={},
 pdfcreator={Emacs 25.3.1 (Org mode 9.3.6)}, 
 pdflang={English}}
\begin{document}

\tableofcontents

---
title: "在org mode中嵌入latex"
date: 2021-02-22T23:30:15+08:00
draft: false
categories: ['life']
tags: ['emacs']
cover: '/img/2021-03-01\textsubscript{latex.jpg}'
---

最近写博客发现就算在想着办法避免使用数学公式,在写有关算法,更不用说数学博客的时候还是不可避免的会碰上好多需要使用latex的情况。
于是干脆在这里记一下怎么在org mode中内嵌latex。


\begin{equation}                        \% arbitrary environments,
x=\sqrt{b}                              \% even tables, figures
\end{equation}                          \% etc

If \(a^2=b\) and \(b=2\), then the solution must be
either $$ a=+\sqrt{2} $$ or \[ a=-\sqrt{2} \].



参考链接:\href{https://orgmode.org/manual/Embedded-LaTeX.html}{官方文档} \href{https://blog.poi.cat/post/spacemacs-plus-org-mode-plus-latex}{Spacemacs和Org-mode和\LaTeX{}}
\end{document}
